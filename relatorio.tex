% filepath: /home/max/ML/TrabalhoPraticoML/relatorio.tex
\documentclass[12pt,a4paper]{article}

% Pacotes
\usepackage[utf8]{inputenc}
\usepackage[brazil]{babel}
\usepackage{graphicx}
\usepackage{amsmath}
\usepackage{amssymb}
\usepackage{hyperref}
\usepackage{float}
\usepackage{booktabs}
\usepackage{caption}
\usepackage{subcaption}
\usepackage{geometry}
\usepackage{listings}
\usepackage{xcolor}

% Configurações de página
\geometry{left=3cm,right=2cm,top=3cm,bottom=2cm}

% Configurações de código
\lstset{
    basicstyle=\ttfamily\small,
    breaklines=true,
    frame=single,
    language=Python,
    commentstyle=\color{gray},
    keywordstyle=\color{blue},
    stringstyle=\color{red}
}

% Informações do documento
\title{Trabalho Prático I - INF01017 \\
\large Modelos Interpretáveis para Previsão de Gravidade de Casos de Dengue}
\author{Gabriel Carvalho Tavares \\ Maximus Borges da Rosa}
\date{
    Universidade Federal do Rio Grande do Sul \\
    Professora: Nome da Professora \\
    \vspace{1cm}
    \today
}

\begin{document}

\maketitle

\begin{abstract}
Este relatório apresenta o desenvolvimento de um modelo de aprendizado de máquina para classificação da gravidade de casos de dengue. O trabalho abrange desde a definição do problema até a avaliação de diferentes algoritmos, incluindo análise exploratória dos dados, pré-processamento, e comparação de desempenho entre modelos.
\end{abstract}

\tableofcontents
\newpage

%==============================================================================
\section{Definição do Problema e Coleta de Dados}
%==============================================================================

Nosso trabalho tem como objetivo desenvolver um modelo baseado em AM para prever a gravidade de um caso de dengue a partir de informações gerais do paciente, dados de residência, características da notificação individual, dados clínicos e dados laboratoriais de pacientes. Para tanto, utilizamos o conjunto de dados do Sistema de Informação de Agravos de Notificação (SINAN), disponibilizado em Base dos Dados Base dos Dados - Conjuntos de dados abertos pelo Ministério da Saúde.

\subsection{Contextualização}
% Descreva o contexto do problema de dengue no Brasil
% Importância da classificação de gravidade
% Justificativa para uso de ML

\subsection{Definição do Problema}
% Problema de classificação multiclasse
% Classes: low_risk, alarm, severe
% Objetivo: predizer gravidade com base em características clínicas

\subsection{Fonte dos Dados}
% SINAN (Sistema de Informação de Agravos de Notificação)
% Período de coleta
% Número inicial de registros
% Descrição das variáveis disponíveis

\subsection{Caracterização do Dataset}
% Número de features
% Tipos de atributos (categóricos, numéricos)
% Distribuição das classes (desbalanceamento)

%==============================================================================
\section{Análise Exploratória}
%==============================================================================

\subsection{Análise Descritiva}
% Estatísticas descritivas das variáveis numéricas
% Tabelas de frequência para variáveis categóricas

\subsection{Distribuição das Classes}
% Gráfico de distribuição das classes
% Discussão sobre desbalanceamento

\begin{figure}[H]
    \centering
    % \includegraphics[width=0.7\textwidth]{figures/class_distribution.png}
    \caption{Distribuição das classes no dataset.}
    \label{fig:class_dist}
\end{figure}

\subsection{Análise de Correlação}
% Matriz de correlação entre variáveis numéricas
% Identificação de features relevantes

\subsection{Análise de Valores Ausentes}
% Percentual de missing values por feature
% Estratégias para tratamento

\subsection{Detecção de Outliers}
% Box plots
% Análise de valores extremos

%==============================================================================
\section{Pré-processamento}
%==============================================================================

\subsection{Tratamento de Valores Ausentes}
% Métodos utilizados (remoção, imputação)
% Justificativa das escolhas

\subsection{Codificação de Variáveis Categóricas}
% One-hot encoding
% Label encoding
% Features categóricas tratadas

\subsection{Normalização e Padronização}
% Normalização de features numéricas
% Método utilizado (StandardScaler, MinMaxScaler)
% Importância da normalização para os algoritmos escolhidos

\subsection{Tratamento de Desbalanceamento}
% Técnicas utilizadas (SMOTE, undersampling, class_weight)
% Justificativa da abordagem escolhida
% Impacto nas distribuições das classes

\begin{table}[H]
    \centering
    \caption{Distribuição das classes antes e após balanceamento.}
    \label{tab:balancing}
    \begin{tabular}{lcc}
        \toprule
        \textbf{Classe} & \textbf{Antes} & \textbf{Após} \\
        \midrule
        Low Risk & X & Y \\
        Alarm & X & Y \\
        Severe & X & Y \\
        \bottomrule
    \end{tabular}
\end{table}

\subsection{Seleção de Features}
% Features removidas e justificativa
% Features mantidas
% Critérios de seleção

%==============================================================================
\section{Definição da Abordagem, Algoritmos e Estratégia de Avaliação}
%==============================================================================

\subsection{Abordagem Geral}
% Pipeline de ML adotado
% Fluxo de trabalho

\subsection{Algoritmos Selecionados}
% Justificativa da escolha dos algoritmos
\subsubsection{Árvore de Decisão}
% Características do algoritmo
% Vantagens e desvantagens
% Hiperparâmetros principais

\subsubsection{Random Forest}
% Ensemble learning
% Redução de variância
% Hiperparâmetros

\subsubsection{Gradient Boosting}
% XGBoost ou LightGBM
% Boosting vs Bagging
% Configurações

\subsubsection{Outros Algoritmos}
% SVM, KNN, Naive Bayes, etc.
% Breve descrição

\subsection{Estratégia de Avaliação}

\subsubsection{Particionamento dos Dados}
% Holdout (70/15/15)
% Conjunto de treino, validação e teste
% Estratificação

\subsubsection{Validação Cruzada}
% K-fold cross-validation
% Leave-One-Out CV (se aplicável)

\subsubsection{Métricas de Avaliação}
% Accuracy
% Precision, Recall, F1-Score
% Macro avg vs Weighted avg
% Justificativa das métricas escolhidas para problema desbalanceado

\subsubsection{Otimização de Hiperparâmetros}
% Grid Search ou Random Search
% Hiperparâmetros tunados
% Métrica de otimização (F1-macro)

%==============================================================================
\section{Spot-checking de Algoritmos}
%==============================================================================

\subsection{Metodologia}
% Descrição do processo de spot-checking
% Configurações iniciais dos modelos
% Baseline de comparação

\subsection{Resultados do Spot-checking}

\subsubsection{Árvore de Decisão}
% Resultados iniciais
% Análise de overfitting/underfitting
% Profundidade da árvore

\begin{table}[H]
    \centering
    \caption{Métricas da Árvore de Decisão no conjunto de validação.}
    \label{tab:dt_metrics}
    \begin{tabular}{lccc}
        \toprule
        & \textbf{Precision} & \textbf{Recall} & \textbf{F1-Score} \\
        \midrule
        Macro Avg & 0.XXX & 0.XXX & 0.XXX \\
        Weighted Avg & 0.XXX & 0.XXX & 0.XXX \\
        \bottomrule
    \end{tabular}
\end{table}

\subsubsection{Random Forest}
% Resultados
% Comparação com árvore única

\subsubsection{Gradient Boosting}
% Desempenho
% Tempo de treinamento

\subsubsection{Outros Algoritmos}
% Resultados dos demais modelos testados

\subsection{Comparação de Desempenho}

\begin{table}[H]
    \centering
    \caption{Comparação de F1-Score (Macro) entre algoritmos.}
    \label{tab:comparison}
    \begin{tabular}{lcc}
        \toprule
        \textbf{Algoritmo} & \textbf{F1-Score (Validação)} & \textbf{Tempo (s)} \\
        \midrule
        Árvore de Decisão & 0.XXX & X.XX \\
        Random Forest & 0.XXX & X.XX \\
        Gradient Boosting & 0.XXX & X.XX \\
        SVM & 0.XXX & X.XX \\
        \bottomrule
    \end{tabular}
\end{table}

\begin{figure}[H]
    \centering
    % \includegraphics[width=0.8\textwidth]{figures/model_comparison.png}
    \caption{Comparação visual de desempenho dos algoritmos.}
    \label{fig:model_comp}
\end{figure}

\subsection{Seleção do Modelo Base}
% Critérios de seleção
% Modelo escolhido para otimização

%==============================================================================
\section{Otimização e Avaliação Final}
%==============================================================================

\subsection{Treinamento Final}
% Retreinamento com treino + validação
% Configuração final do modelo

\subsection{Avaliação no Conjunto de Teste}

\subsubsection{Métricas Gerais}
\begin{table}[H]
    \centering
    \caption{Métricas do modelo final no conjunto de teste.}
    \label{tab:final_metrics}
    \begin{tabular}{lccc}
        \toprule
        & \textbf{Precision} & \textbf{Recall} & \textbf{F1-Score} \\
        \midrule
        Macro Avg & 0.XXX & 0.XXX & 0.XXX \\
        Weighted Avg & 0.XXX & 0.XXX & 0.XXX \\
        \midrule
        \multicolumn{4}{c}{\textbf{Accuracy: 0.XXX}} \\
        \bottomrule
    \end{tabular}
\end{table}

\subsubsection{Métricas por Classe}
\begin{table}[H]
    \centering
    \caption{Desempenho por classe no conjunto de teste.}
    \label{tab:class_metrics}
    \begin{tabular}{lccc}
        \toprule
        \textbf{Classe} & \textbf{Precision} & \textbf{Recall} & \textbf{F1-Score} \\
        \midrule
        Low Risk & 0.XXX & 0.XXX & 0.XXX \\
        Alarm & 0.XXX & 0.XXX & 0.XXX \\
        Severe & 0.XXX & 0.XXX & 0.XXX \\
        \bottomrule
    \end{tabular}
\end{table}

\subsubsection{Matriz de Confusão}
\begin{figure}[H]
    \centering
    % \includegraphics[width=0.6\textwidth]{figures/confusion_matrix.png}
    \caption{Matriz de confusão do modelo final.}
    \label{fig:confusion}
\end{figure}

\subsection{Análise de Erros}
% Análise dos casos mal classificados
% Padrões de erro
% Possíveis causas

\subsection{Interpretabilidade}
% Feature importance (se aplicável)
% Análise de decisões do modelo
% SHAP values (opcional)

%==============================================================================
\section{Sumarização dos Resultados}
%==============================================================================

\subsection{Principais Descobertas}
% Síntese dos resultados obtidos
% Modelo com melhor desempenho
% Métricas alcançadas

\subsection{Limitações do Estudo}
% Limitações dos dados
% Limitações dos modelos
% Desafios encontrados

\subsection{Trabalhos Futuros}
% Possíveis melhorias
% Outras abordagens a serem testadas
% Coleta de mais dados
% Feature engineering adicional

\subsection{Considerações Finais}
% Conclusão geral do trabalho
% Aplicabilidade prática
% Contribuições

%==============================================================================
\section*{Referências}
%==============================================================================
\begin{enumerate}
    \item Scikit-learn Documentation. \url{https://scikit-learn.org/}
    \item Imbalanced-learn Documentation. \url{https://imbalanced-learn.org/}
    \item XGBoost Documentation. \url{https://xgboost.readthedocs.io/}
    \item SINAN - Sistema de Informação de Agravos de Notificação. Ministério da Saúde.
\end{enumerate}


\section{Resultados Detalhados}
%==============================================================================
% Tabelas e gráficos adicionais

\end{document}